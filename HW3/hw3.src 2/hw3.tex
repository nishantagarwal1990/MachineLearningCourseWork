\documentclass[12pt, fullpage,letterpaper]{article}

\usepackage[margin=1in]{geometry}
\usepackage{url}
\usepackage{amsmath,amsthm,amssymb}

\usepackage{pgfplots}
\usepgfplotslibrary{polar}
\usepgflibrary{shapes.geometric}
\usetikzlibrary{calc}

\pgfplotsset{my style/.append style={axis x line=middle, axis y line=
middle, xlabel={$x$}, ylabel={$y$}, axis equal }}

\newcommand{\semester}{Fall 2015}
\newcommand{\assignmentId}{3}
\newcommand{\releaseDate}{Oct 20, 2015}
\newcommand{\dueDate}{Nov 3, 2015}

\newcommand{\bx}{{\bf x}}
\newcommand{\bw}{{\bf w}}

\title{CS 5350/6350: Machine Learining \semester}
\author{Homework \assignmentId}
\date{Handed out: \releaseDate\\
  Due date: \dueDate}

\begin{document}
\maketitle


\input{general-instructions}

\section*{Note}
Do not just put down an answer. We want an explanation. No points will
be given for just a statement of the results of a proof. You will be
graded on your reasoning, not just on your final result. 

Please follow good proof technique; what this means is if you make
assumptions, state them. If what you do between one step and the next
is not trivial or obvious, then state how and why you are doing what
you are doing. A good rule of thumb is if you have to ask yourself
whether what you're doing is obvious, then it's probably not obvious.
Try to make the proof clean and easy to follow.

\section{Warm Up: Feature Expansion}
\label{sec:feature-expansion}

[10 points total] Consider an instance space consisting of points on
the two dimensional plane $(x_1,x_2)$. Let $\mathcal{C}$ be a concept
class defined on this instance space. Each function
$f_r \in \mathcal{C}$ is defined by a radius $r$ as follows:
\[
f_r(x_1, x_2) = 
\begin{cases}
  +1  & \text{if } x_1^2 +x_2^2 - 2x_1 \leq r^2 \\
  -1 & \text{else}
\end{cases}
\]
This hypothesis class is definitely not separable in $\mathbb{R}^2$.
That is, there is no $w_1, w_2$ and $b$ such that $f_r(x_1, x_2) =
sign(w_1 x_1 + w_2 x_2 + b)$ for any $r$. 

\begin{enumerate}
\item ~[4 points] Construct a function $\phi(x_1,x_2)$ that maps
  examples to a new space, such that the positive and negative
  examples are linearly separable in that space? That is, after the
  transformation, there is some weight vector $\bw$ and a bias $b$
  such that $f_r(x_1, x_2) = sign(\bw^T\phi(x_1, x_2) + b)$ for any
  value of $r$.

  (Note: This new space need not be a two-dimensional space.)

\item ~[3 points] If we change the above function to: 
  \[
  g_r(x_1,x_2) = 
  \begin{cases}
    +1 & \text{if } x_1^2 -x_2^2 \leq r^2 \\
    -1 & \text{else}
  \end{cases}
  \]

  Does your $\phi(x_1,x_2)$ make the above linearly separable?  If so
  demonstrate how. If not prove that it does not.

\item ~[3 points] Does $\phi(x_1,x_2) = [x_1,x_2^2]$ make the function
  $g_r$ above linearly separable? If so demonstrate how. If not prove
  that it does not.

\end{enumerate}


%%% Local Variables:
%%% mode: latex
%%% TeX-master: "hw3"
%%% End:


\input{pac-learning}

\section{VC Dimension}
\label{sec:vc-dimension}
\begin{enumerate}
\item ~[5 points] Assume that the three points below can be labeled
  in any way.  Show with pictures how they can be shattered by a
  linear classifier.  Use filled dots to represent positive classes
  and unfilled dots to represent negative classes.


  \begin{tikzpicture}
    \begin{axis}[my style, xtick={-1,0,...,3}, ytick={-1,0,...,3},
      xmin=-1, xmax=3, ymin=-1, ymax=3]
      \addplot[mark=*,only marks] coordinates {(2,2)(1,1)(1,2)};
    \end{axis}
  \end{tikzpicture}
  
\item {\bf VC-dimension of axis aligned rectangles in $\mathbb{R}^d$}:
  Let $H^d_{rec}$ be the class of axis-aligned rectangles in
  $\mathbb{R}^d$. When $d=2$, this class simply consists of rectangles
  on the plane, and labels all points strictly outside the rectangle
  as negative and all points on or inside the rectangle as positive.
  In higher dimensions, this generalizes to $d$-dimensional boxes,
  with points outside the box labeled negative.

  \begin{enumerate}
  \item ~[10 points] Show that the VC dimension of $H^2_{rec}$ is 4.
  \item ~[10 points] Generalize your argument from the previous proof
    to show that for $d$ dimensions, the VC dimension of $H^d_{rec}$
    is $2d$.
  \end{enumerate}
  
\item In the lectures, we considered the VC dimensions of infinite
  concept classes. However, the same argument can be applied to finite
  concept classes too. In this question, we will explore this setting.

  \begin{enumerate}
  \item ~[10 points] Show that for a finite hypothesis class
    $\mathcal{C}$, its VC dimension can be at most
    $\log_2\left(|\mathcal{C}|\right)$. (Hint: You can use
    contradiction for this proof. But not necessarily!)

  \item ~[5 points] Find an example of a class $\mathcal{C}$ of
    functions over the real interval $X = [0,1]$ such that
    $\mathcal{C}$ is an {\bf infinite} set, while its VC dimension is
    exactly one.

  \item ~[5 points] Give an example of a {\bf finite} class
    $\mathcal{C}$ of functions over the same domain $X = [0,1]$ whose
    VC dimension is exactly $\log_2(|\mathcal{C}|)$.

  \end{enumerate}
  
\end{enumerate}

%%% Local Variables:
%%% mode: latex
%%% TeX-master: "hw3"
%%% End:



\end{document}
%%% Local Variables:
%%% mode: latex
%%% TeX-master: t
%%% End:
